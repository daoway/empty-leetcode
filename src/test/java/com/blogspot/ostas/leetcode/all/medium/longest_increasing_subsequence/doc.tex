\documentclass{article}
\usepackage{listings}
\usepackage{fontspec}
\usepackage[margin=1in]{geometry}
\usepackage{amsmath}
\usepackage{xcolor}
\usepackage[bookmarks]{hyperref}
\usepackage{a4wide}

% Set JetBrains Mono as the default monospaced font
%% >choco install jetbrainsmono
\setmonofont{JetBrainsMono-Regular}

\lstdefinestyle{intelliJStyle}{
	language=Java,
	basicstyle=\ttfamily\small,
	numbers=left,
	numberstyle=\tiny,
	stepnumber=1,
	numbersep=5pt,
	frame=single,
	breaklines=true,
	breakatwhitespace=true,
	tabsize=1,
	showstringspaces=false,
	captionpos=b % Place the caption at the bottom
}
\lstset{style=intelliJStyle}

\begin{document}
	\bibliographystyle{plain} % We choose the "plain" reference style

    \title{Dynamic Programming Wonderland}
    \author{Stanislav Ostapenko}
    \date{\today}
    \maketitle

	\begin{abstract}
		Your abstract goes here.
	\end{abstract}

\clearpage

	\tableofcontents % Generate table of contents
	\lstlistoflistings % Generate list of listings

\clearpage % or \newpage

\clearpage

	\thispagestyle{empty}

	\vspace*{\fill}
	\begin{center}
		\Huge
		\begin{align*}
			&   \mathcal{O}(1) &&= \mathcal{O}(\text{yeah})\\
			&    \mathcal{O}(\log_{} n) &&= \mathcal{O}(\text{nice})\\
			&    \mathcal{O}(n) &&= \mathcal{O}(\text{k})\\
			&    \mathcal{O}(n^{2}) &&= \mathcal{O}(\text{my})\\
			&    \mathcal{O}(2^{n}) &&= \mathcal{O}(\text{no})\\
			&    \mathcal{O}(n!) &&= \mathcal{O}(\text{mg})\\
			&    \mathcal{O}(n^{n}) &&= \mathcal{O}(\text{sh*t!})
		\end{align*}
	\end{center}
	\vspace*{\fill}


\clearpage % or \newpage
\section{Longest Increasing Subsequence}

\subsection{Recursive Algorithm}
\begin{lstlisting}[caption=$\mathcal{O}(2^{n})$ Recursive Algorithm]
	public class Solution {
		private int max;
		public int lengthOfLISHelper(int[] arr, int n) {
			// Base case: if there is only one element, the LIS length is 1
			if (n == 1) {
				return 1;
			}
			int currResult;
			int maxEnding = 1;
			
			for (int i = n - 1; i > 0; i--) {
				// Recursively calculate LIS for previous elements
				currResult = lengthOfLISHelper(arr, i);
				
				// Check if the current element can be included in the increasing subsequence
				if ((arr[i - 1] < arr[n - 1]) && (currResult + 1 > maxEnding)) {
					maxEnding = currResult + 1;
				}
			}
			max = Math.max(maxEnding, max);
			return maxEnding;
		}
		
		public int lengthOfLIS(int[] nums) {
			lengthOfLISHelper(nums, nums.length);
			return max;
		}
	}
\end{lstlisting}

\clearpage % or \newpage
\subsection{Backtracking Algorithm}
\begin{lstlisting}[caption=$\mathcal{O}(2^{n})$ Backtracking Algorithm]
	public class Solution {
		private List<List<Integer>> generateSubsequences(int[] arr) {
			List<List<Integer>> allSubsequences = new ArrayList<>();
			generateSubsequencesHelper(arr, 0, new ArrayList<>(), allSubsequences);
			return allSubsequences;
		}
		
		private void generateSubsequencesHelper(int[] arr, int index, List<Integer> current, List<List<Integer>> allSubsequences) {
			if (index == arr.length) {
				// Base case: add the current subsequence to the result
				allSubsequences.add(new ArrayList<>(current));
				return;
			}		
			// Exclude the current element
			generateSubsequencesHelper(arr, index + 1, current, allSubsequences);		
			// Include the current element
			current.add(arr[index]);
			generateSubsequencesHelper(arr, index + 1, current, allSubsequences);		
			// Backtrack to exclude the current element
			current.removeLast();
		}
		
		private boolean isStrictlyIncreasing(List<Integer> list) {
			for (int i = 1; i < list.size(); i++) {
				if (list.get(i) <= list.get(i - 1)) {
					return false;
				}
			}
			return true; // Strictly increasing
		}
		
		public int lengthOfLIS(int[] nums) {
			List<List<Integer>> allSubsequences = generateSubsequences(nums);
			int max = 1;
			for (List<Integer> subsequence : allSubsequences) {
				if (isStrictlyIncreasing(subsequence)) {
					max = Math.max(max, subsequence.size());
				}
			}
			return max;
		}
	}
\end{lstlisting}

\clearpage % or \newpage
    \subsection{Bottom-up Dynamic Programming solution}
    Let's define $L(i)$ as the length of the longest strictly increasing subsequence ending at index $i$.
    The recurrence formula for the longest strictly increasing subsequence is given by:
    \[ L(i) = 1 + \max_{\substack{j < i \\ \text{arr}[j] < \text{arr}[i]}} L(j) \]

    This equation states that the length of the longest increasing subsequence ending at index i is 1 plus the maximum length obtained by considering all indices j less than i, where the corresponding element arr[j] is less than arr[i].
    \\

    \noindent
    Complexity :\\
    \\
    \noindent
    $T(n) = \mathcal{O}(n^{2})$
    \\
    $M(n) = \mathcal{O}(n)$

    \begin{lstlisting}[caption=$\mathcal{O}(n^{2})$ DP solution]
class Solution {
	private int max(int[] L) {
		int maxLength = Integer.MIN_VALUE;
		for (final int length : L) {
			maxLength = Math.max(maxLength, length);
		}
		return maxLength;
	}
	
	public int lengthOfLIS(int[] nums) {
		int n = nums.length;
		int[] L = new int[n];
		// Initialize the array with minimum length 1 for each index
		Arrays.fill(L, 1);
		
		// Iterate to fill in the values of L(i) using the recurrence relation
		for (int i = 1; i < n; i++) {
			for (int j = 0; j < i; j++) {
				if (nums[i] > nums[j]) {
					L[i] = Math.max(L[i], L[j] + 1);
				}
			}
		}
		// Find the maximum value in the array L
		return max(L);
	}
}
    \end{lstlisting}

\clearpage % or \newpage
    \subsection{DP with Binary Search}
    \begin{lstlisting}[caption=$\mathcal{O}(n\log_{}{n})$ DP with Binary Search]
	import java.util.Arrays;

	public class Solution {
		public int lengthOfLIS(int[] nums) {
			if (nums == null || nums.length == 0) {
				return 0;
			}

			int[] dp = new int[nums.length];
			int len = 0;

			for (int num : nums) {
				int index = Arrays.binarySearch(dp, 0, len, num);
				if (index < 0) {
					index = -(index + 1);
				}
				dp[index] = num;
				if (index == len) {
					len++;
				}
			}

			return len;
		}
	}
    \end{lstlisting}


%%\section{Recursive Algorithm}
\section{Fibonacci numbers}

\section{Climbing stairs}

\section{Coins change}

\section{Knapsack}

\section{Longest Common Subsequence}

\section{Shortest Graph Path}

\section{Sort integers by the power value - memoization}

%%\begin{lstlisting}[caption=$\mathcal{O}(2^{n})$ Recursive Algorithm]
%%\end{lstlisting}

% References
\begin{thebibliography}{9}
\bibitem{example1}
Author. (Year). Title. \textit{Journal}, Volume(Issue), Page numbers.

\bibitem{example2}
Another author. (Year). Title. \textit{Conference}, Location, Page numbers.

\end{thebibliography}

\end{document}
