\documentclass{article}
\usepackage[margin=1in]{geometry} % Adjust the margin as needed
\usepackage{amsmath}
\usepackage{listings}

\begin{document}

    \title{Dynamic Programming for dummies}
    \author{Stanislav Ostapenko}
    \date{\today}
    \maketitle

    \section*{300. longest increasing subsequence}
    Let's define $L(i)$ as the length of the longest strictly increasing subsequence ending at index $i$.
    The recurrence formula for the longest strictly increasing subsequence is given by:
    \[ L(i) = 1 + \max_{\substack{j < i \\ \text{arr}[j] < \text{arr}[i]}} L(j) \]

	This equation states that the length of the longest increasing subsequence ending at index i is 1 plus the maximum length obtained by considering all indices j less than i, where the corresponding element arr[j] is less than arr[i].
	
    \lstset{language=Java}
\begin{lstlisting}
class Solution {
	private int max(int[] L) {
		int maxLength = Integer.MIN_VALUE;
		for (final int length : L) {
			maxLength = Math.max(maxLength, length);
		}
		return maxLength;
	}
	
	public int lengthOfLIS(int[] nums) {
		int n = nums.length;
		int[] L = new int[n];
		// Initialize the array with minimum length 1 for each index
		Arrays.fill(L, 1);
		
		// Iterate to fill in the values of L(i) using the recurrence relation
		for (int i = 1; i < n; i++) {
			for (int j = 0; j < i; j++) {
				if (nums[i] > nums[j] && L[i] < L[j] + 1) {
					L[i] = L[j] + 1;
				}
			}
		}
		// Find the maximum value in the array L
		return max(L);
	}
}
\end{lstlisting}

\end{document}
